% Template for PLoS
% Version 3.4 January 2017
%
% % % % % % % % % % % % % % % % % % % % % %
%
% -- IMPORTANT NOTE
%
% This template contains comments intended 
% to minimize problems and delays during our production 
% process. Please follow the template instructions
% whenever possible.
%
% % % % % % % % % % % % % % % % % % % % % % % 
%
% Once your paper is accepted for publication, 
% PLEASE REMOVE ALL TRACKED CHANGES in this file 
% and leave only the final text of your manuscript. 
% PLOS recommends the use of latexdiff to track changes during review, as this will help to maintain a clean tex file.
% Visit https://www.ctan.org/pkg/latexdiff?lang=en for info or contact us at latex@plos.org.
%
%
% There are no restrictions on package use within the LaTeX files except that 
% no packages listed in the template may be deleted.
%
% Please do not include colors or graphics in the text.
%
% The manuscript LaTeX source should be contained within a single file (do not use \input, \externaldocument, or similar commands).
%
% % % % % % % % % % % % % % % % % % % % % % %
%
% -- FIGURES AND TABLES
%
% Please include tables/figure captions directly after the paragraph where they are first cited in the text.
%
% DO NOT INCLUDE GRAPHICS IN YOUR MANUSCRIPT
% - Figures should be uploaded separately from your manuscript file. 
% - Figures generated using LaTeX should be extracted and removed from the PDF before submission. 
% - Figures containing multiple panels/subfigures must be combined into one image file before submission.
% For figure citations, please use "Fig" instead of "Figure".
% See http://journals.plos.org/plosone/s/figures for PLOS figure guidelines.
%
% Tables should be cell-based and may not contain:
% - spacing/line breaks within cells to alter layout or alignment
% - do not nest tabular environments (no tabular environments within tabular environments)
% - no graphics or colored text (cell background color/shading OK)
% See http://journals.plos.org/plosone/s/tables for table guidelines.
%
% For tables that exceed the width of the text column, use the adjustwidth environment as illustrated in the example table in text below.
%
% % % % % % % % % % % % % % % % % % % % % % % %
%
% -- EQUATIONS, MATH SYMBOLS, SUBSCRIPTS, AND SUPERSCRIPTS
%
% IMPORTANT
% Below are a few tips to help format your equations and other special characters according to our specifications. For more tips to help reduce the possibility of formatting errors during conversion, please see our LaTeX guidelines at http://journals.plos.org/plosone/s/latex
%
% For inline equations, please be sure to include all portions of an equation in the math environment.  For example, x$^2$ is incorrect; this should be formatted as $x^2$ (or $\mathrm{x}^2$ if the romanized font is desired).
%
% Do not include text that is not math in the math environment. For example, CO2 should be written as CO\textsubscript{2} instead of CO$_2$.
%
% Please add line breaks to long display equations when possible in order to fit size of the column. 
%
% For inline equations, please do not include punctuation (commas, etc) within the math environment unless this is part of the equation.
%
% When adding superscript or subscripts outside of brackets/braces, please group using {}.  For example, change "[U(D,E,\gamma)]^2" to "{[U(D,E,\gamma)]}^2". 
%
% Do not use \cal for caligraphic font.  Instead, use \mathcal{}
%
% % % % % % % % % % % % % % % % % % % % % % % % 
%
% Please contact latex@plos.org with any questions.
%
% % % % % % % % % % % % % % % % % % % % % % % %

\documentclass[10pt,letterpaper]{article}
\usepackage[top=0.85in,left=2.75in,footskip=0.75in]{geometry}
\usepackage{graphicx}
\graphicspath{ {figures/} }

% amsmath and amssymb packages, useful for mathematical formulas and symbols
\usepackage{amsmath,amssymb}

% Use adjustwidth environment to exceed column width (see example table in text)
\usepackage{changepage}

% Use Unicode characters when possible
\usepackage[utf8x]{inputenc}

% textcomp package and marvosym package for additional characters
\usepackage{textcomp,marvosym}

% cite package, to clean up citations in the main text. Do not remove.
\usepackage{cite}

% Use nameref to cite supporting information files (see Supporting Information section for more info)
\usepackage{nameref,hyperref}

% line numbers
\usepackage[right]{lineno}

% ligatures disabled
\usepackage{microtype}
\DisableLigatures[f]{encoding = *, family = * }

% color can be used to apply background shading to table cells only
\usepackage[table]{xcolor}
\definecolor{Gray}{gray}{0.45}
\definecolor{LightGray}{gray}{0.9}
\definecolor{VeryLightGray}{gray}{0.93}
\definecolor{LightCyan}{rgb}{0.88,1,1}
\definecolor{Goldenrod}{rgb}{0.95,0.95,0.9}

% array package and thick rules for tables
\usepackage{array}

%further packages (not from PLoS template)
\usepackage{csvsimple}
\usepackage{float}
\usepackage[caption = false]{subfig}


% create "+" rule type for thick vertical lines
\newcolumntype{+}{!{\vrule width 2pt}}

% create \thickcline for thick horizontal lines of variable length
\newlength\savedwidth
\newcommand\thickcline[1]{%
  \noalign{\global\savedwidth\arrayrulewidth\global\arrayrulewidth 2pt}%
  \cline{#1}%
  \noalign{\vskip\arrayrulewidth}%
  \noalign{\global\arrayrulewidth\savedwidth}%
}

% \thickhline command for thick horizontal lines that span the table
\newcommand\thickhline{\noalign{\global\savedwidth\arrayrulewidth\global\arrayrulewidth 2pt}%
\hline
\noalign{\global\arrayrulewidth\savedwidth}}


% Remove comment for double spacing
%\usepackage{setspace} 
%\doublespacing

% Text layout
\raggedright
\setlength{\parindent}{0.5cm}
\textwidth 5.25in 
\textheight 8.75in

% Bold the 'Figure #' in the caption and separate it from the title/caption with a period
% Captions will be left justified
\usepackage[aboveskip=1pt,labelfont=bf,labelsep=period,justification=raggedright,singlelinecheck=off]{caption}
\renewcommand{\figurename}{Fig}

% Use the PLoS provided BiBTeX style
\bibliographystyle{plos2015}

% Remove brackets from numbering in List of References
\makeatletter
\renewcommand{\@biblabel}[1]{\quad#1.}
\makeatother

% Leave date blank
\date{}

% Header and Footer with logo
\usepackage{lastpage,fancyhdr,graphicx}
\usepackage{epstopdf}
\pagestyle{myheadings}
\pagestyle{fancy}
\fancyhf{}
\setlength{\headheight}{27.023pt}
\lhead{\includegraphics[width=2.0in]{PLOS-submission.eps}}
\rfoot{\thepage/\pageref{LastPage}}
\renewcommand{\footrule}{\hrule height 2pt \vspace{2mm}}
\fancyheadoffset[L]{2.25in}
\fancyfootoffset[L]{2.25in}
\lfoot{\sf PLOS}

%% Include all macros below

\newcommand{\lorem}{{\bf LOREM}}
\newcommand{\ipsum}{{\bf IPSUM}}

%% END MACROS SECTION


\begin{document}
\vspace*{0.2in}

% Title must be 250 characters or less.
\begin{flushleft}
{\Large
\textbf\newline{Predicting phenotypic traits of microbes from their genome}
}
\newline
% Insert author names, affiliations and corresponding author email (do not include titles, positions, or degrees).
\\
Fred Farrell\textsuperscript{1},
Orkun S. Soyer\textsuperscript{1*},
Christopher Quince\textsuperscript{2*},
\\
\bigskip
\textbf{1} School of Life Sciences, University of Warwick, Coventry, United Kingdom
\\\textbf{2} Warwick Medical School, University of Warwick, Coventry, United Kingdom
\\
%\textbf{3} Affiliation Dept/Program/Center, Institution Name, City, State, Country
%\\
\bigskip

% Insert additional author notes using the symbols described below. Insert symbol callouts after author names as necessary.
% 
% Remove or comment out the author notes below if they aren't used.
%
% Primary Equal Contribution Note
%\Yinyang These authors contributed equally to this work.

% Additional Equal Contribution Note
% Also use this double-dagger symbol for special authorship notes, such as senior authorship.
%\ddag These authors also contributed equally to this work.

% Current address notes
%\textcurrency Current Address: Dept/Program/Center, Institution Name, City, State, Country % change symbol to "\textcurrency a" if more than one current address note
% \textcurrency b Insert second current address 
% \textcurrency c Insert third current address

% Deceased author note
%\dag Deceased

% Group/Consortium Author Note
%\textpilcrow Membership list can be found in the Acknowledgments section.

% Use the asterisk to denote corresponding authorship and provide email address in note below.
* Joint corresponding authors: c.quince@warwick.ac.uk, o.soyer@warwick.ac.uk

\end{flushleft}
% Please keep the abstract below 300 words
\section*{Abstract}

Predicting function from genomic information remains an unresolved problem. One that is particularly relevant in the context of microbial communities research, where we are increasingly able to use metagenomics to obtain novel genomes from a variety of natural and man-made environments such as the ocean, human gut, and biotechnological reactors. Here, we develop and test different machine learning based classifier models to assign functions to metagenome assembled genomes (MAGs). We have trained and cross-validated these models using a database of 84 phenotypic traits linked to 9,407 available, fully-sequenced genomes. We found that different classifier models perform differently and when using different feature vectors. The best model performance resulted when we used a lasso logistic regression model with KEGG ortholog usage frequency as the decision feature. This model was able to classify 65 of these traits with greater than 90\% cross-validated AUROC score. We show the utility of this classifier model by assigning traits to novel MAGs from three different metagenomic studies focusing on different environments.

% Please keep the Author Summary between 150 and 200 words
% Use first person. PLOS ONE authors please skip this step. 
% Author Summary not valid for PLOS ONE submissions.   
%\section*{Author summary}
%Lorem ipsum dolor sit amet, consectetur adipiscing elit. Curabitur eget porta erat. Morbi consectetur est vel gravida pretium. Suspendisse ut dui eu ante cursus gravida non sed sem. Nullam sapien tellus, commodo id velit id, eleifend volutpat quam. Phasellus mauris velit, dapibus finibus elementum vel, pulvinar non tellus. Nunc pellentesque pretium diam, quis maximus dolor faucibus id. Nunc convallis sodales ante, ut ullamcorper est egestas vitae. Nam sit amet enim ultrices, ultrices elit pulvinar, volutpat risus.

\linenumbers

% Use "Eq" instead of "Equation" for equation citations.
\section*{Introduction}
Predicting phenotype from genotype remains one of the major challenges in biology ~\cite{Green2008,Martiny2015}. Addressing this challenge is particularly relevant for understanding microbial communities, the study of which has been boosted by an increased ability to resolve sequence data directly from communities. Technological improvements in DNA sequencing have led to an explosion in the amount of such data generated. In the context of microbial ecology, large-scale metagenomic studies such as the Human Microbiome Project~\cite{Huttenhower2012}, the Earth Microbiome Project~\cite{Gilbert2014} and the Tara Oceans Project~\cite{Zhang2015} have systematically sequenced the microbial communities in a huge variety of environments at great depth. Amplicon sequencing, such as of the 16S rRNA gene, allows detailed study of the taxonomic makeup of these communities, while shotgun metagenomic sequencing allows characterisation of all genes present in an environment. Increasing depth of coverage and improvements in genome binning algorithms for clustering contigs into genomes, in particular the use of differential coverage across different samples~\cite{Alneberg2014, Eren2015}, are allowing more and more full and partial genomes to be assembled from shotgun metagenomic studies. Many of these organisms are novel and uncultured, having never been studied in a lab. A recent metagenomic study on aquifer systems~\cite{Anantharaman2016}, for example, reconstructed 2540 separate high-quality, near-complete genomes, and claimed to have discovered an astonishing 47 new phylum-level lineages among them.

Converting this exponentially growing sequence data into functional understanding of microbial communities requires us to determine physiological functions from it~\cite{Widder2016}. This would allow the inference of key functions in microbial
communities found in human and animal guts, soil, and the oceans, and how these functions
change over ecological conditions and with time~\cite{Widder2016}. In
turn, this ability, could allow us to discern ecological adaptations in environmental
microbial communities, as well as to achieve functional mechanistic models of stability and function~\cite{Louca2016c}. It would be a key step towards managing communities underpinning human and animal health and biogeochemical cycles.


Efforts to achieve phenotype-genotype mapping from environmental sequence data has so far mostly focussed on phylogenetic assignments using the 16S rRNA gene. This highly conserved
gene can provide a phylogenetic assignment at the species (or higher) level, which can then be used to infer general functional traits. While this approach has been commonly used to study ecological distribution of microbial functions e.g.~\cite{Philippot2009,Louca2016a,Louca2016b}, its premise of a direct association of function with phylogenetic assignment (i.e.
‘functional coherence of microbial taxa’) is questionable (e.g.~\cite{Philippot2010}).
The level of taxonomic coherence of function is not clear even for strains of the same species,
where functions can show high variability either due to a few genetic changes or even
regulatory changes~\cite{Martiny2015}; as seen for example among \emph{Escherichia coli} strains ~\cite{Sabarly2011}. It
is also a common problem that when certain taxonomic groups in a microbial community are
found to show direct associations with certain ecological factors (or health state of an host),
these groups are phylogenetically so broad that assigning specific functions to them is hard or impossible~\cite{Koeppel2012}. Indeed, a study of specific functional traits
across microbial taxa has found that many of these traits are dispersed across the
phylogenetic tree~\cite{Martiny2015,Martiny2013}. Even where a specific functional trait is taxonomically coherent, the
phylogenetic approach is limited by our ability to assign taxonomy based on the 16S rRNA gene. The extensive accumulation of metagenomics data indicates that we have sampled only a fraction of microbial diversity, and it is not uncommon for such data to result in many
unassigned taxa, as for example by the aquifer study described above which described dozens of apparently novel phyla.

The scope to perform functional characterisation of environmental samples has expanded with the advent of metagenomics sequencing. This technology has allowed development of several bioinformatics pipelines that can go from raw metagenomic sequence data to predicted (i.e. binned) genomes. These metagenome assembled genomes (MAGs) are then analysed for the presence of specific functional traits (e.g. ~\cite{Narayanasamy2016}). The functional annotation steps in existing bioinformatics pipelines usually considers presence or absence of specific genes in a MAG, and associated with known metabolic pathways, as identified for example in databases such as the Kyoto
Encyclopedia of Genes and Genomes 22 (KEGG)~\cite{Kanehisa2017}, the Pfam database of protein families~\cite{Finn2016} and the NCBI COG database of orthologous genes~\cite{Tatusov1997}. This approach circumvents the
problems of taxonomic coherence and 16S rRNA gene based assignment to taxa, yet relies heavily on existing categorization of genes into functions as done in the above databases. While these
functional gene groupings are mostly based on accumulated knowledge and experimental
data on metabolic pathways, they might miss the full set of genes associated with a given
function and do not consider functions that cannot be assigned to a few genes or seemingly
well-organised pathways. One route to overcome such limitations is to
develop extensive databases of phenotypic traits of microbes without necessarily using a
pathway-centric view. These functional assignments could then serve as a source to apply
statistical approaches to `learn' genetic drivers of those functions
using genomes of associated microbes. Efforts in this direction have recently resulted in
the compilation of literature-based assignment of functions in microbes, either covering a large selection of functions and organisms~\cite{Louca2016,Louca2017} or specific ones such as methanogenesis~\cite{ukaszewicz2015}. The FAPROTAX database~\cite{Louca2016}, which we focus on here, is based on an extensive survey of the scientific literature. 
The aim of creating this database was to allow microbes found from 16S rRNA amplicon sequencing to be assigned into functional and metabolic groups, so that functional variation across environemnts could be studied and compared to taxonomic variation. The authors found that the abundance of functional groups was strongly influenced by environmental conditions in a variety of ocean environments~\cite{Louca2016a}. The bulk of the classifications in the FAPROTAX database come from \emph{Bergey's Manual of Systematic Bacteriology}~\cite{Whitman}, and it currently contains 84 phenotypic traits associated with 4600 microbial taxonomic groups.

Here, we use this accumulating data on phenotypic traits to develop an algorithmic approach to create functional assignments of MAGs. To do this, we combined the FAPROTAX functional trait database with 9407 genomes downloaded from the NCBI, and used machine learning based approaches to train statistical models able to infer an organism’s phenotypic traits from their genome. For model construction and training we compared the use of gene orthologs from the KEGG database and Pfam domains. We show that the resulting classifiers perform better than simple taxonomic assignments of trait, and reveal both known and new genetic drivers of specific traits. Using the resulting classifiers for over 80 traits, we were able to analyse three recent, large-scale metagenomics datasets from three diverse environments: anaerobic digesters, ground water aquifers and the ocean. The classifier-based functional analysis of these datasets revealed significant differences in functional properties between environments and conditions. This work was inspired by a
recent software famework, Traitar~\cite{Weimann2016}, which uses SVMs to predict
microbial traits based on genomic information in the form of copy numbers of Pfam families. Our work differs from Traitar in the use of the highly detailed FAPROTAX database. Traitar utilised the Global Infectious Disease and Epidemiology Online Network
(GIDEON)~\cite{Berger2005} for its phenotypic annotations, and was therefore biased
toward pathogenic traits; we instead focus on traits associated with metabolism and
environmental niche.


\section*{Materials and methods}
\subsection*{Databases and preparation of training data}
To train the models, we utilized the combination of the recently published FAPROTAX database of microbial phenotypes and the NCBI genome database. We downloaded all prokaryotic genomes classified as `full' from the NCBI Genome database. We used the taxonomic information available from NCBI to assign them phenotypes using the script `collapse\_table.py' which comes as part of the FAPROTAX database~\cite{Louca2016}. We then called genes in these genomes using Prodigal~\cite{Hyatt2010}. We annotated the resulting inferred coding DNA sequences (CDS) both by aligning against the KEGG database using Diamond BLASTP~\cite{Buchfink2014} and by searching with hmmer3~\cite{Eddy2011} against Pfam~\cite{Finn2016} all with standard settings. The result is a matrix of organisms and and their copy numbers of either KEGG orthologs or Pfam domains. We have found that using gene copy number rather than simple presence/absence significantly increases classifier performance. The scripts we used to download and process the genomes are available at https://github.com/chrisquince/GenomeAnalysis.git.

\subsection*{Statistical modelling}
To model the link between genotype, in the form of KEGG ortholog copy numbers or Pfam protein families, and phenotypes (i.e. class targets) as represented in the FAPROTAX database, we used a variety of machine learning techniques. These are all different ways of learning the relationship between the features (gene copy numbers) and targets (biological traits) from the training data of 9407 NCBI genomes from unique species. To train the algorithms, we first split this data into a training set (75\% of the genomes) and a test set (25\%), by random sampling. The algorithms were then trained on the training data, and their performance tested on the unseen test data genomes, to check that the relationships learned are generalisable (i.e. that the algorithms have not `overfit' the training data). Below, we describe the algorithms used.

\subsubsection*{Logistic regression}
We found logistic regression, a commonly used linear model for classification problems,~\cite{Hastie2009a,Freedman2009} to be an effective strategy in our case. We scaled all input features to have mean zero and variance one before performing the regression. Since the number of KEGG orthologs (features) was somewhat larger than the number of training examples, overfitting, whereby the model classifies on features of the training set which are very specific to it, was potentially a problem. To alleviate this, we used logistic regression with an $\ell$1 penalty term, also known as LASSO logistic regression~\cite{Lee2006}, whereby parameters are penalized in such a way that only a few of the features have a nonzero weight. In detail, the method involves adding a penalty term equal to the $\ell$1-norm of all of the coefficients of the regressor, thereby penalising nonzero terms, to create an optimisation problem defined by;
\begin{equation}
\min_{w}\left[||w||_1 + C\sum_{i=1}^{n}\log\left(\exp(-y_i(X_i^T w )) + 1\right)\right]
\end{equation}
where $w$ is the vector of regression weights, $X_i$ are the feature vectors of each sample, $y_i$ the classification targets, $n$ is the number of data points, and $C$ is a parameter defining the (inverse) strength of the regularization. This method of regularization is often useful in cases where the number of features is large (similar to or larger than the number of training examples), as most of the features are not used in the classification task. 
%In our case, we found that this method significantly outperformed other commonly-used and %somewhat more complex classification algorithms, such as random forests and support %vector machines.

\subsubsection*{Random forests}
We also used the random forest algorithm, a popular machine learning method which can be applied to both regression and classification problems, which is simple to use, fast and performs fairly well on a wide variety of problems~\cite{Hastie2009}. The random forest is an `ensemble' method, using a collection of slightly randomized classifiers, the results of which are averaged to produce a prediction. This helps to avoid overfitting. A random forest is an ensemble of so-called decision trees. A decision tree is a model which learns to split up training examples into sets according to their feature values, with the aim of separating the target classes. They have the advantage of being invariant under scaling of features and adding of irrelevant features, this last feature being useful in our case where the number of features is very large and many are irrelevant to the classification task; they can also learn more complex relationships between variables than a linear model such as logistic regression. However, an individual tree tends to overfit the training data. A random forest trains a large number of such trees on random subsets of the features and combines these predictions by averaging, avoiding overfitting and much improving performance. 

\subsubsection*{Support vector machines}
Finally, we used support vector machines (SVMs)~\cite{Hastie2009a}. An SVM essentially tries to find surfaces in the high-dimensional feature space which separate the different classes as well as possible, and with as wide a margin as possible between the surface and the examples. These surfaces can be either linear or non-linear (if a non-linear kernel is used); they are therefore capable of capturing complex non-linear relationships between features and targets. They can also include regularization terms as in logistic regression, to reduce overfitting. 


\subsection*{Metrics and classifier performance}
Since many of the classes which we are attempting to predict are highly unbalanced (e.g. of the 9407 unique species with full genomes in the NCBI database only 83 are hydrogentotrophic methanogens), simple classification accuracy is not a very useful measure of classifier performance. Predicting all labels as negative in the above example would give an accuracy of 99.1\% despite not being a useful classifier. We therefore need a metric which can take into account class imbalance. We use the area under the ROC (Receiver Operating Characteristic) curve, which is a graph of true postive rate against false positive rate as one varies the cutoff in probability for making a positive prediction~\cite{Fawcett2006,Flach2011}. An AUROC score much greater than 0.5 (the score for random predictions) indicates a good classifier. In particular, a score of 1 indicates that all positive cases have been assigned a higher probability than all negative cases.

\subsection*{Prediction of MAG phenotypes}
Once classifiers have been trained on the NCBI data, it is possible to use them to make predictions about unseen genomes, such as MAGs generated from shotgun sequencing studies. The MAGs must first be processed to give a matrix of the KEGG ortholog copy numbers associated with them, using the same pipeline as applied above to the NCBI genomes. These matrices are then used as input into the classifiers to produce a matrix of MAGs and their predicted traits, which can be either presence/absence predictions or probabilities. 

\subsection*{MAG collections}
We applied our classifier to three separate collections of MAGs from three different studies:
\begin{itemize}

\item Tara Oceans MAG collection: This comprised a subset of 660 MAGs from the collection 
of 957 non-redundant MAGs generated from the Tara Oceans microbiome in Delmont {\it et al.}~\cite{Delmont2017}. These 660 MAGs were those which had at least 75\% of the 36 single-copy prokaryotic core genes identified in Alneberg {\it et al.} \cite{Alneberg2014} in a single-copy and can thus be considered reasonably complete and pure prokaryotic genomes. The Tara Oceans microbiome survey generated 7.2 terabases of metagenomic data from 243 samples across 68 locations from epipelagic and mesopelagic waters around the globe \cite{Sunagawa2015}, Delmont {\it et al.} extracted their MAGs from a subset of 93 of these samples, 61 surface samples and 32 from the deep chlorophyll maximum layer. Therefore these MAGs represent a substantial sample of planktonic microbial life.

\item Anaerobic digester (AD) MAG collection: This comprised a collection of 153 MAGs that were constructed by co-assembly and binning of 95 metagenome samples taken from three 
replicate laboratory anaerobic digestion (AD) bioreactors converting distillery waste into biogas. They were assembled with \texttt{Ray} using a kmer size of 41 and all 186,081 contig fragments greater than $2 kbp$ in length were clustered by \texttt{CONCOCT} \cite{Alneberg2014} generating a total of 355 bins of which 153 had 75\% of single-copy core genes present in a single copy. These were considered high quality enough to be used in this analysis. 

\item Candidate phyla radiation (CPR) MAG collection: This collection of 581 MAGs is a subset of 797 MAGs provided by the authors of Brown \emph{et al.} 2015~\cite{Brown2015}. They comprise members 
of the Candidate phyla radiation (CPR) assembled from ground water enriched with acetate.

\end{itemize} 

% Results and Discussion can be combined.
\section*{Results}
\subsection*{Classifiers based on genomic features achieve over 90\% accuracy in assigning phenotypic traits}

To create a phenotypic trait assignment for MAGs, we have created different machine learning classifiers and that take as input genomic features.  Specifically, we build classifiers based on genes and their copy numbers (see Methods). Genes were identified from the genome using existing bioinformatics pipelines and the databases of KEGG orthologs (KOs) ~\cite{Kanehisa2017} or Pfam gene families \cite{Finn2016}. Using the resulting gene and copy number information as ‘features’ of a genome, we then trained classifiers on these features to assign specific phenotypic traits to those genomes. We used as a training platform the known phenotypic traits, as compiled in the FAPROTAX database ~\cite{Louca2016}, and over 9000 fully annotated genomes, as listed on the NCBI database. We used this general approach to develop different machine learning based classifiers (see Methods).  

To evaluate the performance of the different classifiers, we used the Area Under Receiver Operating Characteristic (AUROC), a common measure to evaluate classifiers (see Methods). The AUROC scores were calculated using k-fold cross-validation with $k=5$, i.e. the data was split into training and testing sets 5 times, in such a way that each training example was in the test set once, and the prediction for each data point when it was in the test set was used.  The results are shown in Figure~\ref{fig1} for the three classification algorithms, $\ell$1-regularized logistic regression, the random forest and a linear SVM. The regularized logistic regression outperforms the random forest in assigning most, though not all, of the phenotypic traits assessed.
The average score over all traits for logistic regression is 90.1\% (versus 84.5\% for the random forest), and 65 traits have a score greater than 90\%, with 45 higher than 95\%. The perfmormance of the SVM and the logistic regression are similar, although they do differ significantly for some traits. The mean score of the SVM is slightly better, at 90.8\% vs. 90.1\%. This difference is not statistically significant (paired t test, p=0.71), and since logistic regression is easier to interpret and much more computationally efficient, we decided to focus on it for the rest of the paper.

% Place figure captions after the first paragraph in which they are cited.
\begin{figure}[!h]
\includegraphics[width=0.9\textwidth]{fig1}
\caption{{\bf Overall performance of classification algorithms.}
The AUROC score on each classification task (each trait) is shown for three classification algorithms: $\ell$1-regularized logistic regression, the random forest and a linear support vector machine (SVM). Traits are ordered by the logistic regression score.}
\label{fig1}
\end{figure}

Additionally, we can compare the results we obtain using the two different ways of obtaining genes (and gene copy numbers) from genomes, namely KOs and Pfam families, see Figure ~\ref{orth_comp}. The results using the two schemes are rather similar, though there are some traits where one outperforms the other; this may reflect better coverage of the genes involved in the trait in a particular scheme. On average, KO performs better, with a mean score of 90.1\% versus 84.9\% for Pfam ($p<0.001$), and we therefore concentrate on the approach of obtaining genes from genomes using KOs for the rest of the analysis presented below.

It is instructive to compare the performance of the classifiers to the use of KEGG modules, where an equivalent module exists for that trait, i.e. compare the performance to a `classifier' where an organism is judged capable of a trait if it has a complete KEGG module for that trait. Table~\ref{tab2} shows the results of this comparison for three FAPROTAX traits with corresponding KEGG modules. Note that the KEGG module method does not require training, so the metrics are over the entire NCBI dataset, whereas for the classifier they are only for the held-out test set. Also, the former method gives only presence/absence of a trait rather than a probability, so the AUROC score cannot be calculated, so we use alternative metrics based on classification: the $F_1$ score and the confusion matrix.

It can be seen that the logisitc regression classifier does significantly better than KEGG modules in assigning these traits as they appear in the FAPROTAX database. This suggests that having the enzymes or proteins described in the KEGG module for a trait is not in fact a necessary or sufficient condition for actually performing that trait, and that other genes are more predictive. However, it is possible that the discrepancy is due instead to inaccuracy in the FAPROTAX database, e.g. species which do perform the traits being missed from the database and therefore getting flagged as false positives with the KEGG method. More work would be needed to fully exclude this possibility.  

\begin{table}
\scriptsize
\begin{tabular}{|c|c|c|c|c|}\cline{2-5}%

 \multicolumn{1}{c|}{} &  \multicolumn{2}{|c|}{\bfseries KEGG modules} & \multicolumn{2}{|c|}{\bfseries Classifier} \\ \hline
 \rowcolor{LightGray} \bfseries module & \bfseries F1 & \bfseries confusion matrix & \bfseries F1 & \bfseries confusion matrix \\\hline
%\rowcolor{VeryLightGray}
sulfate respiration & 0.84 & $\begin{pmatrix}9197 & 53 \\ 6 & 151\end{pmatrix}$ & 0.99 & $\begin{pmatrix}2313 & 0 \\ 1 & 38\end{pmatrix}$\\ \hline
%\rowcolor{VeryLightGray}
nitrate respiration & 0.14 & $\begin{pmatrix}6821 & 2162\\ 224 & 200\end{pmatrix}$ & 0.622 & $\begin{pmatrix}2237 & 9 \\ 54 & 52\end{pmatrix}$\\ \hline
%\rowcolor{VeryLightGray}
hydrogenotrophic methanogensis & 0.756 & $\begin{pmatrix}9281 & 43 \\ 6 & 77\end{pmatrix}$ & 0.923 & $\begin{pmatrix} 2331 & 0\\ 3 & 18\end{pmatrix}$\\ \hline
\end{tabular}

\caption{{\bf Comparison of classifiers to KEGG modules.}
Table showing the performance of using KEGG module presence/absence against logistic regression classifiers for some traits where equivalent KEGG modules exist. Since the KEGG module approach does not give a probability, the AUROC score cannot be used, so the F1 score and confusion matrices are compared.}\label{tab2}
\end{table}

\subsection*{Classifier performance is not due to taxonomic coherence}

It is possible that the classifiers are simply picking genes that are irrelevant to a specific trait, but rather are present in taxonomically close species that all happen to perform a given trait. To explore this possibility, we first looked at how traits spread over the taxonomy. We found that this varies between traits (Figure~\ref{fig4}), but as expected, organisms in the same taxa usually present similar traits. 

To investigate this phenomenon and attempt to find orthologs with real causal associations with traits, we tried training a model on one taxa and testing its performance on the others. If a classifier can predict phenotype based on genes in a distantly-related, unseen set of organisms, it is likely the genes it is using have a real association with the trait. In particular, we tried training the logistic regression models on the Proteobacteria, a large phylum of bacteria, and testing on the rest of the taxonomy. Some traits did not have significant numbers of species in each of these sets; we used only traits with at least 5 species in the training set and 5 in the test, leaving 59 traits out of 84. 

\begin{figure}
\includegraphics[width=0.45\textwidth]{sulfate_resp_tree}
\includegraphics[width=0.45\textwidth]{nitrate_resp_tree}
\caption{{\bf Taxonomic distribution of metabolic traits.}
Taxonomic trees of all prokaryotic NCBI species with full genomes. For training the cross-taxa verison of the classifier, only the Proteobacteria (red section of the tree) were used, and the models were tested on the rest of the tree. Species capable of a) sulfate respiration and b) nitrate respiration are highlighted on the trees.}
\label{fig4}
\end{figure}

As might be expected, this `enforced' ignoring of taxonomic information reduced the performance of the classifiers, compared to being trained on a random selection of species from throughout the prokaryotic part of the tree of life, see Figure~\ref{fig3}b. However, for a significant number of traits the performance of the classsifier is still fairly good, indicating an ability to make predictions which are generalizable to significantly different unseen groups of organisms. 19 traits have an AUROC score greater than 80\%, and 9 greater than 90\%. 

Figure~\ref{fig3}b shows a scatter plot of classifier complexity against performance for the models trained on Proteobacteria only. Compared to the use of all taxa in training, Figure~\ref{fig3}a, it is notable that the group of classifiers achieving high accuracy while using a lot of genes is gone: traits such as fermentation and nitrate reduction, which were in this group of classifiers, are now much less accurate. Classifiers which work well in the cross-taxa case all use a relatively small number of genes, less than 150 or so. This suggests that the classifiers using a large number of genes to make predictions in the randomized case may have been using a range of genes found in different closely-related clusters of organisms which all have the target trait, but which may not have a causal relationship with the trait.


\subsection*{Resulting classifiers highlight the importance of specific and non-obvious gene combinations in defining trait assignments}

The classifiers discussed above are built upon genomic features, in this case KO gene frequencies. Thus, we can analyse each classifier for each phenotypic trait and determine which genes underpin those traits. Table~\ref{tab1} gives the KOs that have non-zero coefficients in the resulting logistic regression classifiers for some example traits, for which the classifier had an AUROC score above 95\%. Many of the KOs involved are those known to be involved in the associated trait, as one might have hoped. In particular, consider the prediction of methanogens, a relatively easy task since it is known that methanogens must possess the \emph{mcrA} gene, this being a necessary and sufficient condition for methanogenesis~\cite{Deppenmeier2002}. Indeed, subunits of this gene have the highest weight, and a total of only 9 genes are used by the classifier. 

Looking at some more complex traits, for example sulfate respiration (i.e. disimmilatory sulfate reduction to H$_{2}$S), the model assigns a lot of weight to subunits of a quinone-modifying oxidoreductase, which is indeed associated with sulfur metabolism~\cite{Chan2009}. Interestingly, however, none of the genes picked out by the classifier are directly part of the metabolic pathway for this process as described in the KEGG module for dissimilatory sulfate reduction, see Figure~\ref{fig2}. The situation is similar with hydrogenotrophic methanogenesis, with classification mostly determined by components of energy-converting hydrogenases which are not directly part of the autotrophic methanogenesis pathway, along with \emph{mcr} genes indicating that the microbe is a methanogen.  

Given this observation that different classifiers make use of different numbers of KOs, we further explored the relation between number of features (KO identity) used in a classifier and its performance (see Figure~\ref{fig3}). This analysis demonstrates that there is indeed a correlation between these two variables, with some highly accurate classifiers built out of a large number of genes. However, there is also a noticable cluster of traits with high accuracy achieved with only a few genes (less than 100). These traits may be particularly interesting, as it is more likely that these small groups of orthologs are causally associated with the trait, rather than just being genes which typically occur in parts of the phylogenetic tree which have the trait and may or may not have any direct relation to it. This issue is explored further in the section on performance across taxa, below. Also, note that most of the traits that perform poorly, which typically use very few genes to classify, have very low support in the training data in terms of number of positive examples.

The ability of the classifiers to assign traits based on few and specific KOs shows that genomes with a specific trait must be clustering in the high-dimensional KO space. To explore this possibility, we have created an ordination plot (Figure~\ref{tsne}) of all the species in the training and test sets using their KEGG ortholog copy numbers. That is, a dimensionality reduction algorithm, here stochastic neighbour embedding, has been applied to visualise variation in all KO copy numbers in two dimensions. Points are colored by whether they are true positive, true negative, false positive or false negatives under a particular classification task, here for sulfate respiration. It can be seen that the species performing this trait do tend to cluster together into a few groups in the KO space, allowing the algorithm to classify them mostly correctly. 

\begin{table}
\scriptsize
\begin{tabular}{|r|l|c|}\hline%
\rowcolor{Goldenrod}
\multicolumn{3}{|c|}{\bfseries Sulfate respiration} \\ \hline
\bfseries KO & \bfseries Weight & \bfseries Description\\\hline
\csvreader[late after line=\\\hline]%
{datafiles/sulfate_respiration_kos.csv}{ko=\ko,weight=\weight,DEFINITION=\defin}%
{\ko & \weight & \defin}%
\label{tab1}
\end{tabular}
\begin{tabular}{|r|l|c|}\hline%
\rowcolor{Goldenrod}
\multicolumn{3}{|c|}{\bfseries Methanogenesis} \\ \hline
\bfseries KO & \bfseries Weight & \bfseries Description\\\hline
\csvreader[late after line=\\\hline]%
{datafiles/methanogenesis_kos.csv}{ko=\ko,weight=\weight,DEFINITION=\defin}%
{\ko & \weight & \defin}%
\label{tab1}
\end{tabular}
\begin{tabular}{|r|l|c|}\hline%
\rowcolor{Goldenrod}
\multicolumn{3}{|c|}{\bfseries Hydrogenotrophic methanogenesis} \\ \hline
\bfseries KO & \bfseries Weight & \bfseries Description\\\hline
\csvreader[late after line=\\\hline]%
{datafiles/hydro_methanogenesis_kos.csv}{ko=\ko,weight=\weight,DEFINITION=\defin}%
{\ko & \weight & \defin}%


\end{tabular}

\caption{{\bf Details of classifiers for specific traits.}
Tables showing all the nonzero weights in the logistic regression models trained on three traits from the FAPROTAX database. Note that there are 9647 KEGG orthologs used in the models, so the vast majority of weights are set to zero in these models.}\label{tab1}
\end{table}

\begin{figure}
\includegraphics[width=0.6\textwidth]{fig2}
\caption{{\bf KEGG modules for some traits.}
Representations of the KEGG modules corresponding to the FAPROTAX traits shown in Table~\ref{tab1}. Modules are organized into `blocks' of orthologs, typically indicating a protein complex. Orthologs positioned next to each other are `options', i.e. that section of the module is present if any of the adjacent blocks are present.}
\label{fig2}
\end{figure}

\begin{figure}
\subfloat{\includegraphics[width=0.48\textwidth]{fig3}}
\subfloat{\includegraphics[width=0.48\textwidth]{fig6}}
 \caption{Scatterplots showing the AUROC score of the different classifiers plotted against the number of gene orthologs the classifier uses to make its predictions. Point size is proportional to the number of positive examples in the training set. Left: in the standard case. Right: in the cross-taxa case with training on Proteobacteria only.}\label{fig3}
\end{figure}


\subsection*{Trained classifiers can be applied successfully to analyse the phenotypic traits of MAGs}
A major aim of trainig these classifiers is to explore the functional capabilities of novel genomes isolated from metagenomic studies. We therefore used the classifiers trained above to classify metagenomically assembled genomes (MAGs) from a few different environments. These were laboratory anaerobic digesters, the ocean (from the Tara oceans project~\cite{Zhang2015}), and MAGs from a groundwater aquifer assigned to be members of the so-called `canditate phyla radioation' (CPR)~\cite{Anantharaman2016}. The CPR is a set of bacterial lineages discovered from metagenomic studies consisting of a very large number of proposed novel phyla. These organisms have very small genomes, and are suggested to live in symbiosis with other organisms~\cite{Danczak2017}.

To perform the functional assigments, we used the $\ell$1-regularized loigistic regression classifier described above, with a random train-test splitting and the regularization parameter $C=0.05$, trained using KEGG orthologs on the full NCBI genomes. Figure~\ref{heatmap} shows a heatmap with presence or absence of the different traits for the MAGs assembled from anaerobic digesters and from the global oceans (the Tara project). 

There are noticeable differences between traits assigned to MAGs from AD vs. ocean data, as shown in Figure~\ref{mag-compare} (and also Figure S3). Some of these differences reflect obvious differences in the two environments (e.g. aerobic traits missing from the AD data), while others point to more subtle differences that need further exploration.

\begin{figure}
\includegraphics[width=0.9\textwidth]{mag_heatmaps_cpr}
\caption{{\bf Heatmap of presence/absence of fucntions in MAGs.}
Results of running the set of logistic regression classsifiers trained on NCBI genomes on MAGs assembled from three environments: laboratory anaerobic digesters, the ocean and `candidate phyla radiation' (CPR) organisms from an aquifer system.}
\label{heatmap}
\end{figure}


These differences in trait might well be expected between these environents. For example, fermentation is very important in the AD process, and aerobic chemoheterotrophy obviously is not as the environment is aerobic. This indicates that the method is acapable of producing useful information about MAGs. The results for the CPR MAGs indicate that these organisms possess significantly fewer traits than those from the other two environments, as would be expected from their very small genome sizes. A few traits do however have significant incidence in this group. Apart from `chemoheterotrophy' and `aerobic chemoheterotrophy', which are very broad categories encompassing a large proportion of all organisms, a few traits associated with nitrogen metabolism, especially nitrate reduction, are noticeably present in the group. That the CPR are involved in nitrate reduction was recently proposed in Danczak et al. ~\cite{Danczak2017}.

Some of the functional assignments seem strange, for example organisms being classified as acetoclastic or hydrogenotrophic methanogens but not as methanogens, including a significant proportion of CPR organisms (about 3\%), which are bacteria, being classified as acetoclastic methanogens. Looking at the gene orthologs present in these organisms and their taxonomic assignments sheds some light on what is going on here, see Table~\ref{tab3}. For example, some of the acetoclastic methanogens which are misclassified as not being methanogens are missing the \emph{mcrA} ortholog K00399, presumably because the MAGs are incomplete and this gene has been missed. Another example is an organism classified as being a hydrogenotrophic methanogen but not a methanogen. This MAG appears to be from the bacterium \emph{Caldisericum exile}, which is not a methanogen and does not possess \emph{mcrA} (it is an anaerobic, thermophilic bacterium which respires by thiosulfate reduction~\ref{Mori2009b}). However, it does possess genes for subunits of the energy-converting hydrogenase A, noted earlier (Table~\ref{tab1}) to be incidative of hydrogenotrophic methanogenesis. Therefore, these discrepancies may be the result either of incomplete MAGs, or of combinations of genes which are rare or unseen in the training set.

For the Tara dataset, metadata for different samples was available. Combining the coverages of the different MAGs across these samples with the functional assignments of the MAGs, we can calculate the proportion of total coverage which comes from microbes with a given trait. Figure~\ref{oceans} shows the mean of this metric for all the traits for samples from different oceans. The differences do not seem to be very significant between oceans, with a few exceptions, notably an abundance of nitrate reduction in the Southern Ocean. Samples metadata included depth, temperature and salinity, among other measurements. Figure~\ref{corr_tara} shows the correlation between temperature and fermenter abundance, showing a rather large negative correlation (Pearson correlation $r=-0.43, p<0.001$).

\begin{table}
\scriptsize
\csvautotabular{datafiles/methanogen_mag_details.csv}

\caption{{\bf Key genes and predicted traits for MAGs predicted to be methanogens.}
Gene copy numbers for the \emph{mcrA} methanogenesis gene and the energy-converting hyrdogenase A, along with functional predictions, for AD MAGs predicted to be methanogenic by our algorithm.}\label{tab3}
\end{table}

\begin{figure}
\includegraphics[width=0.98\textwidth]{oceans}
\caption{{\bf Mean abundance of microbes performing traits by ocean.}
Average of the proportion of total coverage associated with microbes possessing a trait over samples from each of the oceans.}
\label{oceans}
\end{figure}

\begin{figure}
\includegraphics[width=0.9\textwidth]{temp_corr_tara}
\caption{{\bf Association of temperature and fermenter abundance.}
Scatterplot of proportion of total coverage associated with microbes performing fermentation versus mean temperature of a sample. Pearson $r=-0.43, p<0.001$.}
\label{corr_tara}
\end{figure}


\section*{Discussion}
We have demonstrated a method for inferring phenotypes from genotypes, in the form of gene ortholog copy numbers, using machine learning on an existing phenotype database combined with NCBI genomes. While the accuracy of the predictions vary significantly over different phenotypes, a significant proportion of the traits we tested achieved very good classification accuracy, with AUROC scores greater than 90\%. Of the machine learning algorithms we used, we found that $\ell1$-regularized logistic regression gave the best combination of accuracy, computational efficiency and interpretability.
The results did not depend very strongly on whether KEGG orthologs or Pfam domains were used to characterise genes in the genomes, although the KEGG scheme performed slightly better on average over the traits we considered here. The logistic regression models we generate can be inspected, and the genes most associated with a given phenotype in the model enumerated, which allowed for validation of the models by comparison to what is known about the trait of these genes, and potentially for this method to discover new orthologous groups associated with phenotypes. For example, we found here that the presence of subunits of the energy-converting hydrogenase A are more predictive of an organism's performing hydrogenotrophic methanogensis than the genes directly involved in the process as described in the KEGG functional module for it. For nitrate reduction, in addition to expected genes such as nitrate reductase, there are multiple KEGG orthologs listed as `uncharacterised protein' which are highly predictive of the function.

To check the robustness of the models we generated, we tried training the models on one section of the microbial taxonomic tree, the Proteobacteria, and testing its accuracy on organisms from the rest of the tree, which it had not encountered at all in training. This did significantly reduce model accuracy for many phenotypes. This is to be expected, as the training and test sets in this case are so different. However, some of the traits still achieved good accuracy. This would suggest that the logistic regression model is identifying genes functionally involved with the phenotype in the training stage, such that their presence even in distantly related organisms is indicative of the presence of the phenotype. Phenotypes with good accuracy in this schema tended to produce models involving only a few genes (i.e. only a few genes had nonzero weights in the logistic regression model), less than 100, supporting the idea that they the models are picking out genes directly involved with the phenotype. %Functions which achieved good accuracy in the standard training-testing schema by using a large number of genes had much reduced accuracy in this case.

A major use we envisage for this method is the prediction of phenotypes of novel, uncultivated organisms discovered through metagenomic studies. To demonstrate this, we ran the trained models on collections of metagenomically assembled genomes (MAGs) from three environments: laboratory anaerobic digesters (ADs), the oceans (from the Tara oceans sequencing project), and `candidate phyla raditation' MAGs from an aquifer system. There were significant differences in the predictions in the different environments, and these made sense in terms of what is known about the environments. For example, AD MAGs had high numbers of fementers and low numbers assigned to `aerobic chemoheterotrophy'. CPR MAGs had relatively few traits assigned overall, which makes sense given their very small genome size, but had a relatively large number of nitrate reducers \cite{Danczak2017}. Therefore, these trained models can help us to gain insights into the functional capabilities both of microbiomes as a whole and the individual species making up those communitites, even when the species in these samples are novel and uncultivated.


\section*{Supporting information}
% Include only the SI item label in the paragraph heading. Use the \nameref{label} command to cite SI items in the text.

\subsection*{Supplementary figures}

\renewcommand{\thefigure}{S\arabic{figure}}

\setcounter{figure}{0}

\begin{figure}[!h]
\includegraphics[width=0.9\textwidth]{orth_compare}
\caption{{\bf Performance of classification algorithms using different ortholog schemes.}
The AUROC score on each classification task (each trait) is shown for algorithms trained on two representations of genomes in terms of orthologous groups of genes, the KEGG orthology (KO) and Pfam protein families.}
\label{orth_comp}
\end{figure}

\begin{figure}
\subfloat{\includegraphics[width=0.95\textwidth]{tsne}}\\
\subfloat{\includegraphics[width=0.95\textwidth]{tsne_top_only}}
\caption{{\bf Stochastic neighbour embedding of all species in KEGG ortholog space}
Ordination in two dimensions of all the species in the training dataset. By coloring by classification group (true positive, true negative, false positive, false negative) for a particular trait, here sulfate respiration, we can graphically visualise the behaviour of the classifier. Top: embedding performed over all KEGG orthologs. Bottom: embedding performed only over KOs relevant to the trait according to the classifier.}
\label{tsne}
\end{figure}

\begin{figure}
\includegraphics[width=0.9\textwidth]{mag_comparison_bars_cpr}
\caption{{\bf Overall comparison of AD, Tara and CPR MAGs.}
Proportions of MAGS from the environments having a trait, for some of the most common traits.}
\label{mag-compare}
\end{figure}



\nolinenumbers

% Either type in your references using
% \begin{thebibliography}{}
% \bibitem{}
% Text
% \end{thebibliography}
%
% or
%
% Compile your BiBTeX database using our plos2015.bst
% style file and paste the contents of your .bbl file
% here. See http://journals.plos.org/plosone/s/latex for 
% step-by-step instructions.
% 

\clearpage
\bibliographystyle{plos2015}
\bibliography{trait_prediction,others}{}

% \begin{thebibliography}{10}

% \bibitem{bib1}
% Conant GC, Wolfe KH.
% \newblock {{T}urning a hobby into a job: how duplicated genes find new
%   functions}.
% \newblock Nat Rev Genet. 2008 Dec;9(12):938--950.

% \bibitem{bib2}
% Ohno S.
% \newblock Evolution by gene duplication.
% \newblock London: George Alien \& Unwin Ltd. Berlin, Heidelberg and New York:
%   Springer-Verlag.; 1970.

% \bibitem{bib3}
% Magwire MM, Bayer F, Webster CL, Cao C, Jiggins FM.
% \newblock {{S}uccessive increases in the resistance of {D}rosophila to viral
%   infection through a transposon insertion followed by a {D}uplication}.
% \newblock PLoS Genet. 2011 Oct;7(10):e1002337.

% \end{thebibliography}



\end{document}

